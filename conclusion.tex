\section{Conclusion and Future Work}


In order to meaningfully deal with disparate building streams in a scalable 
fashion the streams should be \emph{searchable} across various properties, such
as building name, room location, and type.
Searchability is necessary for achieving scalability.  By providing a tool for
searching across building streams, we minimize the deployment time for applications that 
allowing them to be used in \emph{all} buildings, not just a single one.  
We describe how a set of programming by example techniques can be used to
learn how to transform a building's metadata 
to a common namespace by using a small number of examples from an expert. 

In order to adapt synthesis techniques presented in prior work~\cite{Gulwani:2011} 
we have to overcome
three fundamental challenges: 1) attaining full tag coverage for a building
is difficult and the number of tags necessary to attain full coverage is very large.
2) merging the substring rules for each tag is non-trivial and 3) because there are so
many points, visual inspection of correctness is very hard.
Our adaptation partially overcomes these challenges and we show how the tag expansion results can
be applied across many building.
%We show how our approach makes it easier to write applications across buildings by
%demonstrating its use by two different applications: 1) a rogue zone detector, and 
%2) an application that identifies and ranks the most comfortable
%rooms. We illustrate these on a testbed consisting of nearly 60 buildings comprising more 
%than 16,000 sense points.

%\texttt{Arka: expand the contribution here and summarize the results}
 %First, the number of tags required to fully qualify an entire building might be large, wheras certain tags may be applicable on a very limited number of sensor names. In such a case, the {\bf Match}$(v_i,r,c)$ expression for each tag should be expressive enough to differentiate a small group of sensor names from the remaining. 

%Second, care should be taken while merging the {\bf SubString} rules for each tag. 

%Second, a building may have 1000s of sensor points, making visual inspection of correctness of sensor name qualification very hard. Incorrectly qualification of sensor names can be mitigated by being more conservative with the boolean preciates in the top level {\bf Switch} operator. 

%what's next? future work
For future work we look to integrate standard feature extraction techniques to enrich the metadata
with semantically descriptive terms that can be used to search for points of interest based on
deeper attributes embedded in the data itself.  Moreover, we plan to integrate more buildings
into the metadata suite to cover a large fraction of building and enable wide development
of analytics and applications.  We also look to explore how our current technique could be applied in the 
wider internet of things context, as more sensors streams are deployed in the home environment.
We believe that such metadata boosting and term indexing technique are necessary to make sense of
the explosion of data coming from sensors.  Buildings present a major challenge and surely the solutions
in this space can be applied in other domains.


