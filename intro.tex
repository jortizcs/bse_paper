\section{Introduction}

Buildings are sites of very large sensor deployments, typically containing
up to several thousand sensors reporting physical measurement, continuously.
Moreover, with the recent interest in reducing building energy consumption, it
is important to consider ways to quickly bootstrap a set of building data streams
into an anlytical pipeline to determine where there are opportunities for energy savings,
discovery of broken sensors, and assessing and tracking overall building performance.
However, current `point' naming conventions form a bottleneck in the scalability of
the data integration process.  A `point' refers to a physical location where
a sensor is taking measurements. Each building vendor uses their own naming scheme and
uniques variants of each scheme are implemented from building to building; variations exist
even across buildings that have contracted the same vendor to set up their deployment.
This makes the integration process laborious and fundamentally unscalable.  If we
wish to have a broad impact across the entire building stock, at scale, we need
to explore methods for overcoming this challenge.

Consider a simple analysis whcih has the ability
to identify anomalous readings from a specific kind of sensor.  In order to run the application
the deployer needs to know the name of the sensorand how to attain readings from it.
The stream identification process is manual.  The deployer loads the interface to the 
building management system, tracks down the spatial or system view, clicks through several windows
to locate the location of the point(s) of interest, mouses over the point(s) and records the name,
and then uses that point name to request it from the data-fetch protocol -- typically BACNet or 
LonTalk or another protocol. This process is repeated in \emph{every building} where this 
application runs.  Any application that uses building data requires access to the building
management system and the network carrying the data of interest.

In order to meaningfully deal with disparate building data streams in a scalable 
fashion the streams should be \emph{searchable} across various properties, such
as building name, room location, and statistical trends.  Moreover, we
assert that wide searchability is necessary for achieving scalability.  By providing a tool for
searching across building streams, we minimize the deployment time for applications that 
allowing them to be used in \emph{all} buildings, not just a single one.  The aforementioned 
building manamgent system user interface implicitly groups sensors by location in space
or association with a system.  This grouping is also captured in the name of the point used by
the underlying communication protocol.  For example, 'AHU' -- air handling unit -- is typically
embedded in the name of every sensor that is associated with a particular air handling unit.
A similar convention is used for denoting the type of data produced by the point (i.e. all points
that contain 'ART' (area room temperature)  in their name refer to a temperature sensor).

We observe that every naming scheme looks to capture three point attributes: 
1) the location in space, 2) its relation to an subsytem, and 3) the type of 
measurement it is taking.  

We want to make the streams searchable.  How do we do that?
1) We need index the metadata for the streams but the metadata available is not enough
2) We need to expand the metadata, but how?
3) name expansion --> tag unification
4) timeseries feature extraction --> tag unification

top things to expand upon:  location, type, system
secondary: statistical features about the data
 
