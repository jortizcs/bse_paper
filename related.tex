\section{Related Work}

Our work borrows from from the literature in the context of providing a way to search
through data from sensors in buildings.  Our approach consists of two main components:
metadata boosting and search.  Our main focus in this paper is on metadata boostting.
We use the technique introduced by Gulwani et al.\cite{Gulwani:2011} in order to learn 
the expanded form of `point` tags.  In their work, and related extensions~\cite{Harris:2011,
Singh:2012,Gulwani12spreadsheetdata}, they provide a set of libraries that implement
algorithms to learn the various patterns for strings in excel spreadsheets.  Because many
excel users are not programmers, it was clear that the interaction model should be based
on having the user provide input-output examples and for a the system to iteratively learn
the pattern the user would write if they were more technical inclined.  In our work
we use a similar approach and make non-trivial extensions to it in order to boost the
existing metadata with a normalized set of tags.  This allow users to broadly search across
all points for multiple buildings at once, without having to engage in the tedious
pattern adjustment task for expansion.

The de facto approach in buildings is to form communities that follow are particular standard.
Each of the vendors follows their own general structure in point naming, however there
are general standards emerging to unify these across vendors, such as the Building
Information Model (BIM)~\cite{BIM} and Green Building XML~\cite{GBXML}.  These describe an 
object structure to describe, in great detail, the various physical components of the building,
the internal subsystems, and the sensors within them.  The standard was mainly proposed
for use by architectural design firms to construct and share building models across software
suites.  
