\section{Related Work}

Our work borrows from from the literature in the context of providing a way to search
through data from sensors in buildings.  Our approach consists of two main components:
metadata boosting and search.  Our main focus in this paper is on metadata boostting.
We use the technique introduced by Gulwani et al.\cite{Gulwani:2011} in order to learn 
the expanded form of `point` tags.  In their work, and related extensions~\cite{Harris:2011,
Singh:2012,Gulwani12spreadsheetdata}, they provide a set of libraries that implement
algorithms to learn the various patterns for strings in excel spreadsheets.  Because many
excel users are not programmers, it was clear that the interaction model should be based
on having the user provide input-output examples and for a the system to iteratively learn
the pattern the user would write if they were more technical inclined.  In our work
we use a similar approach and make non-trivial extensions to it in order to boost the
existing metadata with a normalized set of tags.  This allow users to broadly search across
all points for multiple buildings at once, without having to engage in the tedious
pattern adjustment task for expansion.

The de facto approach in buildings is to form communities that follow are particular standard.
Each of the vendors follows their own general structure in point naming, however there
are evolving astandards that unify these across vendors, such as the Building
Information Model (BIM)~\cite{BIM} and Green Building XML~\cite{GBXML}.  These are used to
codify an object structure to describe the various physical components of the building.
For example, there are objects for the internal subsystems, teh walls, the construction
if the windows, and sensors within the building as well, among other things.  Each
Building software vendor extends or modifies their own BIM version.
These standards are maily 
for use by architectural design firms to construct and share building models across software
suites. We could potentially make use of BIMs in incorporating the descriptive names/tags
as extra metadata.  We leave this exercise for future work.

Google is incorporating 3D models of buildings and embedding their into Google 
Earch 3D Buildings offering~\cite{google_3dbuildings}, making them searchable according 
to their metadata and location.
The effort does not include a granularity down to the sensors, but with the recent acquistion
of NEST~\cite{nest}, indexing the metadata that describes the physical measurement points
within buildings may come next.  Our approach is to normalize the metadata for the sensors,
so that we can maximize coverage and do analysis across many buildings at once.  Because standards
are not followed, boosting is critical to achieving normalization and maximizing coverage.

% smap, fiap, lontalk, bacnet. 
A different approach entiring are those taken by open standards such as BacNet and 
LonTalk~\cite{bacnet, lontalk} and more recent approaches to describe the sensors more
systematically, such as sMAP~\cite{smap} or FIAP~\cite{fiap}.  From a metadata perspective, they
essentially bypass the normalization issue.  We use them as meta/data sources but we address a
fundamental data integration problem to achieve wider coverage, faster.
